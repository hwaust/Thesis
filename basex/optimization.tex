\section{Integration with Query Optimization}
\label{sect:opt}

As mentioned in Section~\ref{sect:basex}, BaseX is equipped with a powerful
query optimizer. Some queries can be optimized to reduce the execute
time.  For example, BaseX optimizes XM3 to
\begin{lstlisting}
/site/open_auctions/open_auction/bidder[last()]
\end{lstlisting}
on the basis of the path index, which brings knowledge that \Src{open\_auction}
exists only immediately below \Src{open\_auctions} and \Src{open\_auctions}
exists only immediately below \Src{site}. Because a step of descendant-or-self
axis (\Src{//open\_auction}) is replaced with two steps of child axes
(\Src{/open\_auctions/open\_auction}), the search space of this query has been
significantly reduced. Note that a more drastic result is observed in XM2 that

, where
the attribute index is exploited through function \Src{db:attribute}. 

Partitioning strategies convert a given query to two separate ones and therefore
affects the capability of BaseX in query optimization. In fact, the suffix query
of XM3(b) is not optimized to the corresponding part of optimized XM3 because
BaseX does not utilize indices for optimizing queries starting from nodes
specified with PRE values even if possible in principle. Most index-based
optimizations are limited to queries starting from the document root. This is a
reasonable design choice in query optimization because it is expensive to check
all PRE values observed. However, we do not have to check all PRE values that
specify the starting nodes of the suffix query because of the nature of data
partitioning, of which BaseX is unaware. This discord between BaseX's query
optimization and data partitioning may incur serious performance degradation, 
and it also occurs in query partitioning strategy in term of parallelizing subquereis.

A simple way of resolving this discord is to apply partitioning strategies after
BaseX's query optimization. Partitioning strategies are applicable to any
multi-step XPath query in principle. Even if an optimized query is thoroughly
different from its original query as in XM2, it is entirely adequate to apply
both partitioning strategies to the optimized query, forgetting the original. In fact,
XM2--4(c) are instances of such data partitioning after optimization.

The simplicity of this coordination brings two big benefits. One is that we are
still able to implement partitioning strategies only by using BaseX's dumps of
optimized queries without any modification on BaseX. The other is that it is
very easy to implement partitioning strategies into compilation in BaseX; we can
just add a data/query-partitioning pass after all existing query optimization
passes without any interference.

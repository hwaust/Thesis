
%: ----------------------- cover page back side ------------------------  
\newpage
\vspace{10mm}
1. Reviewer: 

\vspace{10mm}
2. Reviewer:

\vspace{10mm}
3. Reviewer:

\vspace{10mm}
4. Reviewer:

\vspace{10mm}
5. Reviewer:


\vspace{40mm}
Day of the defense: Febrary 25, 2018.

\vspace{20mm}\hspace{70mm}

Signature from head of PhD committee:



%: ----------------------------- abstract -----------------------------
\begin{abstracts}
	
	In recent decades, the volume of information increases dramatically, leading to
	an urgent demand for high-performance data processing technologies. XML document
	processing as a common and popularly used information processing technique has
	been intensively studied.
	
	About 20 years ago at the very early stage of XML processing, studies mainly
	focused on the sequential ways,  which were limited by the fact that CPUs at
	the time were commonly single-core processors. In the recent decade, with the
	development of multiple-core CPUs, it provides not only more cores we can use
	in a sinlge CPU, but also better availability with cheaper prices.  Therefore,
	parallization become popular in information processing.
	
	Parallelization of XPath queries over XML documents became popular started from
	the recnet decade. At the time, studies focused on a small set of XPath queires
	and were designed to process XML documents in a shared-memory environment.
	Therefore, they are not practical for processing large XML documents, making
	them difficult to meet the requirements of processing rapidly grown large XML
	documents.
	
	To overcome the difficulties, we first revived an existing study proposed by
	Bordawekar et al. in 2008. Their work was implemented on an XSLT processor Xalan
	and has already been out of date now due to the developments of hardware and
	software. We presented our three implementations on top of a state-of-the-art
	XML databases engine BaseX over XML documents sized server gigabytes. Since
	BaseX provides full support for XQuery/XQuery 3.1, we can harness this feature
	to process subqueries from the division of target XPath queries.
	
	This pre-hand study establishes the availability of Bordawekar et al's work.
	Then, we propose a fragmentation approach that divides an XML document into
	node-balanced subtrees with randomization for achieving better load-balance.
	Combined with the previous data partitioning strategy, we can process large XML
	documents efficiently in distribute-memory environments.
	
	The previous partition and fragmentation based study enables us to easily
	process large XML documents in distribute-memory environments. However, it
	still has its flaw that is limited to top-down queries. Therefore, to
	enrich the expressiveness of our study, we then proposed a novel tree, called
	partial tree. With partial tree, we can make the XML processing support more
	types of queries, making it more feasible to process large XML documents 
	by utilizing computer clusters. We also propose an efficient BFS-array based 
	implementation of partial tree.
	
	There are two important contributions proposed in the thesis.
	
	The first contribution involves three implementations of Bordawekar et al's
	partitioning strategies, and our observations and perspectives from the
	experiment results. Our implementations are designed for the parallelization of
	XPath queries on top of BaseX. With these implementations, XPath queries can be
	easily parallelized by simply rewriting XPath queries with XQuery expressions.
	We conduct experiments to evaluate our implementations and the results showed
	that these implementations achieved significant speedups over two large XML
	documents. Besides the experiment results, we also present significant
	observations and perspectives from the experiment results. Then, based on them,
	we extend the fragmentation algorithms to exploit data partitioning strategy in
	distributed-memory environments.
	
	The second contribution is the design of a novel tree structure, called partial
	tree, for parallel XML processing. With this tree structure, we can split an XML
	document into multiple chunks and represent each of the chunks with partial
	trees. We also design a series of algorithms for evaluating queries over these
	partial trees. Since the partial trees are created from separated chunks, we can
	distribute these chunks to computer clusters. In this way, we can run queries on
	them in distributed memory environments. Then, we propose an efficient
	implementation of partial tree. Based on indexing techniques, we developed an
	indexing scheme, called BFS-array index along with grouped index. With this
	indexing scheme, we can implement partial tree efficiently, in both memory
	consumption and absolute query performance. The experiments showed that the
	implementation can process 100s GB of XML documents with 32 EC2 computers. The
	execution times were only seconds for most queries used in the experiments and
	the throughput was approximately 1 GB/s.
	
\end{abstracts}

%: -------------------------- front matter ---------------------------
\frontmatter

% % Thesis Dedictation ---------------------------------------------------

\begin{dedication} %this creates the heading for the dedication page

{\Huge Dedication }\\
\vspace{10mm}

{\LARGE
To Wenjun Xie,

my amazing wife, 

who accompanied me 

through the most difficult time in my life,

and brought me two beautiful and lovely sons.

My parents, Wenlin Hao and Qinglin Zhu,

who supported and encouraged me 

throughout my Ph.D. career.

}

\end{dedication}

% ----------------------------------------------------------------------

% Thesis Dedication ------------------------------------------------
\begin{dedication} %this creates the heading for the dedication page
	
	{\Huge Dedication }\\
	\vspace{10mm}
	
	{\LARGE
		To Wenjun Xie,
		
		my amazing wife, 
		
		who accompanied me 
		
		through the most difficult time in my life,
		
		and brought me two beautiful and lovely sons.
		
		My parents, Wenlin Hao and Qinglin Zhu,
		
		who supported and encouraged me 
		
		throughout my Ph.D. career.
		
	}
	
\end{dedication}


% Thesis Acknowledgements ------------------------------------------------


\begin{acknowledgements}  


There were many persons who provided me a lot of assistance with this work. 
Without their assistance, I could not finish this thesis. Therefore, 
I would like to give my sincere
gratitude to them, particularly the following professors, classmates, family
members, friends etc.

First and foremost, I would like to give my sincerest gratitude to my doctoral
supervisor Assoc. Prof. Kiminori Matsuzaki. It is my greatest honor to be his
first Ph.D. student. Assoc. Prof. Kiminori Matsuzaki is a very kind and amiable
person with the consistent solid support both on my Ph.D. research work and
daily life in Japan. I appreciate all his contributions of energetic enthusiasm,
immense knowledge and experience on research, insightful ideas, and generous
support, making my Ph.D. experience productive and fruitful. I am also thankful
to his excellent advice and examples he has provided as a successful computer
scientist and professor.

% Prof. Li, Jingzhao, who gave me great help in my daily life.

I would like to express sincere appreciation to Dr.Shigeyuki Sato for his great
help on my research work. I have quite often been enlightened by his quite strict
attitude towards research work and setting such a good example for me.

I would like to think Assoc. Prof. Tomoharu Ugawa for his helpful advice 
on my research. I would like to thank Prof. Jingzhao Li, who assisted me 
with some job issues back in China. 

I would like to thank my wife WenJun Xie, my father Wenlin Hao, and my
mother Qinglin Zhu for their significant supports and encouragements.

I would like to thank the following friends: Onofre Call Ruiz, who was my 
classmate and lab mate. He gave me a great help in English learning and 
daily life. I would like to thank Naudia Patterson, who is an English 
teacher and assisted me with revising my thesis.

I would like to express sincere appreciation to Kochi University of Technology
for providing me such a great research opportunity.


\end{acknowledgements}


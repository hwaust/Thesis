\begin{abstracts}

XML processing is an important technique and has been intensively studied. In
recent decades, information is increasing dramatically in size, leading to an
urgent demand for high-performance data processing technologies for large volume
of data, especially in the field of large XML document processing.

Parallelization of XPath queries over XML documents has been popularly studied
in the past decade. Most of these studies either focused on a small set of XPath
queires or were not practical for large XML documents. Thus these studies cannot
meet the requirements of the rapid growth of XML documents.

To overcome the difficulties, we first revived an existing study proposed by
Bordawekar et al. in 2008. Their work was implemented on an XSLT processor Xalan
and has already been out of date now due to the developments of hardware and
software. We presented our three implementations on top of a state-of-the-art
XML databases engine BaseX over XML documents sized server gigabytes. Since
BaseX provides full support for XQuery/XQuery 3.1, we can harness this feature
to process subqueries from the division of target XPath queries. For processing
larger XML documents, we proposed a novel tree, called partial tree. With
partial tree, we extend the processing of XML documents from shared-memory
environments to distributed-memory environments, making it possible to utilize
computer clusters. We also propose an efficient BFS-array based implementation
of partial tree.

There are three important contributions proposed in the thesis.

The first contribution involves three implementations of Bordawekar et al's
partitioning strategies, and our observations and perspectives from the
experiment results. Our implementations are designed for the parallelization of
XPath queries on top of BaseX. With these implementations, XPath queries can be
easily parallelized by simply rewriting XPath queries with XQuery expressions.
We conduct experiments to evaluate our implementations and the results showed
that these implementations achieved significant speedups over two large XML
documents. Besides the experiment results, we also present significant
observations and perspectives from the experiment results.

The second contribution is the design of a novel tree structure, called partial
tree, for parallel XML processing. With this tree structure, we can split an XML
document into multiple chunks and represent each of the chunks with partial
trees. We also design a series of algorithms for evaluating queries over these
partial trees. Since the partial trees are created from separated chunks, we can
distribute these chunks to computer clusters. In this way, we can run queries on
them in distributed memory environments.

The third contribution is an efficient implementation of partial tree. Based on
indexing techniques, we developed an indexing scheme, called BFS-array index
along with grouped index. With this indexing scheme, we can implement partial
tree efficiently, in both memory consumption and absolute query performance. The
experiments showed that the implementation can process 100s GB of XML documents
with 32 EC2 computers. The execution times were only seconds for most queries
used in the experiments and the throughput was approximately 1 GB/s.

\end{abstracts}

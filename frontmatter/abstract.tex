\begin{abstracts}

In recent decades, the volume of information increases dramatically, leading to
an urgent demand for high-performance data processing technologies. XML document
processing as a common and popularly used information processing technique has
been intensively studied.

About 20 years ago at the very early stage of XML processing, studies mainly
focused on the sequential ways,  which were limited by the fact that CPUs at
the time were commonly single-core processors. In the recent decade, with the
development of multiple-core CPUs, it provides not only more cores we can use
in a sinlge CPU, but also better availability with cheaper prices.  Therefore,
parallization become popular in information processing.

Parallelization of XPath queries over XML documents became popular started from
the recnet decade. At the time, studies focused on a small set of XPath queires
and were designed to process XML documents in a shared-memory environment.
Therefore, they are not practical for processing large XML documents, making
them difficult to meet the requirements of processing rapidly grown large XML
documents.

To overcome the difficulties, we first revived an existing study proposed by
Bordawekar et al. in 2008. Their work was implemented on an XSLT processor Xalan
and has already been out of date now due to the developments of hardware and
software. We presented our three implementations on top of a state-of-the-art
XML databases engine BaseX over XML documents sized server gigabytes. Since
BaseX provides full support for XQuery/XQuery 3.1, we can harness this feature
to process subqueries from the division of target XPath queries.

This pre-hand study establishes the availability of Bordawekar et al's work.
Then, we propose a fragmentation approach that can divide an XML document into
size-balanced subtrees with randomization for achieving better load-balance.
Along with the previous data partitioning strategy, we can process large XML
documents efficiently in distribute-memory environments.

The previous partition and fragmentation based study enables us to easily
process  large XML documents in distribute-memory environments. However, it
still has its  flasw that it is limited to top-down queries. Therefore, to
enrich the expressness  of our study, we then proposed a novel tree, called
partial tree. With partial tree, we can make the XML processing support more
types of queries, making it more feasuable to utilize computer clusters. We also
propose an efficient BFS-array based implementation of partial tree.

There are two important contributions proposed in the thesis.

The first contribution involves three implementations of Bordawekar et al's
partitioning strategies, and our observations and perspectives from the
experiment results. Our implementations are designed for the parallelization of
XPath queries on top of BaseX. With these implementations, XPath queries can be
easily parallelized by simply rewriting XPath queries with XQuery expressions.
We conduct experiments to evaluate our implementations and the results showed
that these implementations achieved significant speedups over two large XML
documents. Besides the experiment results, we also present significant
observations and perspectives from the experiment results. Then, based on them,
we extend the fragmentation algorithms to exploit data partitioning strategy in
distributed-memory environments.

The second contribution is the design of a novel tree structure, called partial
tree, for parallel XML processing. With this tree structure, we can split an XML
document into multiple chunks and represent each of the chunks with partial
trees. We also design a series of algorithms for evaluating queries over these
partial trees. Since the partial trees are created from separated chunks, we can
distribute these chunks to computer clusters. In this way, we can run queries on
them in distributed memory environments. Then, we propose an efficient
implementation of partial tree. Based on indexing techniques, we developed an
indexing scheme, called BFS-array index along with grouped index. With this
indexing scheme, we can implement partial tree efficiently, in both memory
consumption and absolute query performance. The experiments showed that the
implementation can process 100s GB of XML documents with 32 EC2 computers. The
execution times were only seconds for most queries used in the experiments and
the throughput was approximately 1 GB/s.

\end{abstracts}

\chapter{Conclusion and Future Work}

This thesis has investigated the parallelization of XPath queries on large XML
documents with two approaches, BaseX and partial tree. We conclude our thesis in
this chapter.

\section{Conclusion}

Parallelization of XPath queries on XML documents has been studied in the past
decade. Most of these studies either focused a small set of XPath queires or
were not practical for large XML documents. Thus these studies cannot meet the
requirements of the rapid grow of XML documents.

To overcome the difficulties, we first revived an existing study proposed by
Bordawerker et al. in 2008. Their work was implemented on Xalan, which is a XSLT
processor and has already been out of date now because the hardware and software
have both changed. We presented our three implementations on top of a
state-of-the-art XML databasex engine BaseX over XML documents sized server
gigabytes. Since BaseX provides full support for XQuery/XQuery 3.1, we can
harness this feature to process subqueries from the division of target XPath
queries.

Through our implementations, we are the first to experimentally prove that it is
possible to obtain significant speedups by simply rewriting queries into
subqueries and parallelizing the evaluation of them on top of an XML database
engine over gigabytes of XML documents, without need to modify source code of
the engine. From the experimental evaluation, our implementations exhibited a
great advantage that we are able to use off-the-shelf XML database engines to
parallelize the evalaution of XPath queries over gigabytes XML documents, which
is very convinient and practical.

For processing larger XML documents, we proposed a novel tree, called partial
tree. With partial tree, we extend the processing of XML documents from
shared-memory environments to distributed-memory environments, making it
possible to utilize computer clusters. We also proposed an efficient BFS-array
based implementation of partial tree. The experiment results showed the
efficiency of our framework that the implementation are able to process 100s GB
of XML documents with 32 EC2 computers. The execution times were only seconds
for most queries used in the experiments and the throughput was approximately 1
GB/s. There is only one known study that reached faster throughput than ours,
which was 2.5 GB/s with 64 cores. However, ours can support more complicated
queries, such as order-aware queries, thus making our approach more expressive.

Besides the throughput, partial tree also has the following two good features.
First, it is practical to evenly divide an large XML document and create partial
trees out of the similar size so that we can reach good load-balancing. Second,
partial trees are portable in both shared-/distributed- memory environemts. This
means that we can make them work in both distributed-memory environments and
shared-memory environments, without changing the setting of partial tree.
Therefore, partial tree is a promising data structure and helpful for parallel
XML processing, especially for large XML documents in distributed-memory
environments.

\section{Future Work}

Based on the studies of BaseX and the partial tree, there are three works that
are worthing doing in the future.

Firstly, our implementations on top of BaseX were evaluated with only a single
BaseX server on a dual-CPU system. Thus, it would be very promising to use
multiple BaseX servers on multiple CPUs in distributed-memory environments over
large XML documents. Considering the PRE values that are integers and very
small, it is suitable to represent the intermediate results of queries by PRE
values to be transmitted among BaseX processors. With more BaseX servers running
on a computer cluster, it is feasible to efficiently process larger XML
documents.

Secondly, although partial trees are suitable for processing XPath queires over
large XML documents, the functionality and fault tolerance, which are both
important for process large XML documents, are still weak when partial trees
work alone. Therefore, developing a partial tree based framework that cooperates
with the distributed systems such as MapReduce or similar frameworks would be a
good designing choice. Also, equipping additional programming interfaces to
handle more complicated queries or larger data is practically important for the
framework.

Lastly, the application of partial tree to BaseX would also be an interesting
work. By the introduction of partial tree into BaseX, we can exploit good
features of partial tree, such as handling imbalanced trees. In this way, it is
high likely to achieve good scalability, especially in case of the
implementation of BaseX in distributed-memory environments.
